\documentclass{article}

\usepackage[margin=1in]{geometry}
\usepackage{graphicx}
\usepackage{amsmath}
\usepackage{amsfonts}


\author{Zachary Vogel}
\date{\today}
\title{Notes in ECEN 5448}

\begin{document}
\maketitle


\section*{Bureaucrucy}
she will be gone Thursday through Sunday\\
office hours: Friday (11am-12pm) in ECEE 1b85\\

\section*{Unstable Zeros}
sometimes digital controllers will have unstable zeros that don't exist in the original system.\\

\section*{Difference Equations}
\[u_k=u_{k-1}+\text{area of tapezoid}\]

\section*{Analysis of Systems}
Transfer functions, state-space representations, block diagram manipulation,\\

\subsection*{Transfer Function}
Z-transform is directly analagous to the Laplace transform\\

Z-transform of difference equation and solve for $\frac{U(z)}{E(z)}$\\

$u_{k-1}\to z^{-1}U(z)$
\[H(z)=\frac{b(z)}{a(z)}\]
solutions of $b(z)=0$ are zeros, solutions of $a(z)=0$ are poles.\\


special case where $H(z)=z^{-1}$ all constant except $b_1$ are zero. This is a unit delay, $u_k=e_{k-1}$.\\

state space, converting from nth order difference equation to n 1st order difference equations.\\

remember that state-space representations are not unique.\\

Controllable canonical form.\\

if you only use simple delays, gains and summers in a state-space rep, you will have a uniquely corresponding state-space representation.\\

look at the notes online for an example of converting high order difference equations to first order ones.\\

feedback loop transfer function, plant $H_1$ and positive feedback $H_2$. Transfer function is:
\[\cfrac{U}{E}=\cfrac{H_1}{1-H_1H_2}\]
so negative feedback would be:
\[\cfrac{U}{E}=\cfrac{H_1}{1+H_1H_2}\]

moving nodes across blocks.\\
for multi-path, multi-loop block diagrams can be simplified using this or Mason's rule.\\

stability\\

for now, only bibo stab\\

bibo stab test in notes, good stuff.\\


\end{document}
