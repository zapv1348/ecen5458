\documentclass{article}

\usepackage[margin=1in]{geometry}
\usepackage{graphicx}
\usepackage{amsmath}
\usepackage{amsfonts}


\author{Zachary Vogel}
\date{\today}
\title{Notes in ECEN 5448}

\begin{document}
\maketitle


\section*{Labs getting started}
progress report on the 7th\\
homework 4 is due March 9th.\\
exam on the 16th.\\

\section*{Lecture}
frequency response is hard to sketch by hand cause the $e^{j\omega T}$ term is akward.\\
review system types, 0, 1, 2, and l.\\

Gain and phase margin defined the same for discrete time as continuous time.\\

gain margin tells you where the system becomes unstable on root locus.\\

\[1+KG(z)=0\]

phase of $G(s)-\frac{\omega T}{2}$ is close to the discrete phase margin.\\
usually $\omega T<\frac{\pi}{2}$.\\

rule of thumb, $PM>0$ and $GM>1$ stable. Not always correct.\\

Nyquist stab crit.\\
evaluate the RHP countour for $D(s)G(s)$ N=zeros-poles, Z is the number of zeros of 1+DG in RHP, and P is poles of 1+DG in the RHP.\\
N is the number of CW encirclements of -1 for DG.\\
If P=2 you want N=-2, so Z=0, which is stable.\\

nyquist for discrete time, zeros outside the unit circle.\\
$1+D(z)G(z)$.\\
N - net CW enciclements of -1,\\
Z - number of zeros outside unit circle.\\
P - number of poles outside unit circcle.\\
the countour around the upper half of the unit circle is the bode plot.\\
branches out to infinity are complex conjugates of each other.\\

alternate criterion because it is more difficult to do the fancy keyhole mapping.\\
just flip the criterion on its head.\\
consider the stable region instead of the unstable one. CCW circle around the unit circle.\\
net ccw encirclements of -1 is the net number of stable zeros to stable poles.\\
keep P and Z as the poles and zeros outside the unit circle.\\
N=P-Z.\\



\end{document}
