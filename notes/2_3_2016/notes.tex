\documentclass{article}

\usepackage[margin=1in]{geometry}
\usepackage{graphicx}
\usepackage{amsmath}
\usepackage{amsfonts}


\author{Zachary Vogel}
\date{\today}
\title{Notes in ECEN 5448}

\begin{document}
\maketitle


\section*{HW1}
89.3 mean, 92 median, std 9.7\\
only a few problems graded carefully\\
main grader comment. Please organize homework so that it is easy to follow and grade. Poor organization gives a reduction in points.\\
scores will be uploaded onto d2l, but there are issues currently.

\section*{Presentation}
paired presentation is about 10-14 minutes total.


\section*{Lecture}
s plane pole locations vs z-plane pole locations.\\
exponentially decaying sinusoid stuff.\\
s plane poles at $s_1,s_2$ then z plane will be at $e^{-s_1 T},e^{-s_2 T}$.\\
look at making a conformal map from the s plane to the z plane.\\
mapping from s to z is many to one.\\
step response characteristics given the poles and zeros of the system.\\
most systems can be described by second order and first order systems in superposition.\\
dominate poles in z-plane are closest to the unit circle.\\

minimum phase systems sometimes go backwards.\\

Rest of the day, sample and hold circuit. Sampling, zero-order hold, first-order hole.\\
what people call an A to D is usually a sample and hold circuit plus and A to D.\\


\[r*(t)=\sum_{k=-\infty}^\infty r(t)\delta (t-kT)\]
Z transform of this is:
\[R(z)=\sum_{k=-\infty}^\infty r(kT)z^{-k}=\sum r(kT)e^{-skT}\]
so $z=e^{sT}$.\\
for $r*(t)=\delta(t)$ all you need is a value of 1 at $t=0$ and a value of zero at all other sample points.\\

sometimes you might want a first order hold instead of a zero order hold.\\
causul first order hold is where you take the current and previous sample and extrapolate that forward.\\
The question is, do you ever to a feed forward type hold?\\

in homework, will compute the trasnfer function of a first order hold.\\
look at slides in homework to do this.\\

Spectrum of a sampled signal.\\
going to sampling theorem, want to reconstruct $r(t)$ perfectly from $r*(t)$.\\
sifting property.\\

\[R*(s)=\cfrac{1}{T}\sum_{n=-\infty}^\infty R(s-jn\omega_s)\]
cut the frequency of your signal, so that there are not overlaps that mess up your wave.\\
sampling frequency must be twice the highest frequency in the signal to reconstruct perfectly.\\
ideal low pass filter in the time domain is:
\[l(t)=sinc\cfrac{\pi t}{T}\]
with $sinc(x)=\cfrac{\sin(x)}{x}$.\\



\end{document}
