\documentclass{article}

\usepackage[margin=1in]{geometry}
\usepackage{graphicx}
\usepackage{amsmath}
\usepackage{amsfonts}


\author{Zachary Vogel}
\date{\today}
\title{Notes in ECEN 5448}

\begin{document}
\maketitle


\section*{Discrete Time Systems}
largest pole inside unit circle, stable difference equation.\\
unit step function here is $1(k)$\\
look at the useful summations thing on web.\\
poles and zeros are the same as with the Laplace transform except z is the variable.\\

Z-transform properties, linearity.\\
convolution\\
time-shift\\
time invariance (shift the input, output will be shifted same amount), this doesn't really work well for sampled systems unless you shift by a multiple of the sample rate.\\
scaling property\\
Final-value theorem\\
methods of inverse transform.\\
\[F(z)=\cfrac{b_0+b_1z^{-1}+\dots+b_mz^{-m}}{1+a_1z^{-1}+a_2z^{-2}+\dots+a_nz^{-n}}\]
can get to the form after this by multiplying by $\frac{z^n}{z^n}$.\\
\[=\cfrac{z^{n-m}(b_0z^m+b_1z^{m-1}+\dots+b_m)}{z^n+a_1z^{n-1}+a_2z^{n-2}+\dots+a_n}\]
then you can use partial fraction expansion to get it in a nice form.\\
inverse z transform integral exists.\\


\end{document}
