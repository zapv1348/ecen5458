\documentclass{article}

\usepackage[margin=1in]{geometry}
\usepackage{graphicx}
\usepackage{amsmath}
\usepackage{amsfonts}


\author{Zachary Vogel}
\date{\today}
\title{Notes in ECEN 5448}

\begin{document}
\maketitle


\section*{Lecture}
nyquist stability continued.\\

Designing compensators from bode plots.\\
crossover frequency and phase margin\\

for compensator, you can back out where the pole needs to be to effect a given frequency with:
\[\omega=\cfrac{1-z_1}{T}\]
breakpoints at:
\[\lvert 1-z_1\rvert=\omega T\]
\[\lvert 1-p_1\rvert=\omega T\]
very accurate if $\omega T\leq 0.1$ sampling frequency is about 60X the breakpoint frequency you want.\\
works till $\omega T\leq 8$ is 45 degrees.\\

lead used to improve transient response, lag used to improve steady-state characteristics.\\

usually for lag compensators, the pole is really close to 1 and the zero is close to 1 as well.\\

margin command for gain and phase margins.\\


\end{document}
