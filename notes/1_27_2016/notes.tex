\documentclass{article}

\usepackage[margin=1in]{geometry}
\usepackage{graphicx}
\usepackage{amsmath}
\usepackage{amsfonts}


\author{Zachary Vogel}
\date{\today}
\title{Notes in ECEN 5448}

\begin{document}
\maketitle


\section*{Lecture}
Homework 2 is due 2 weeks from now, start early. You did shit on the first one.\\

BIBO stability\\

Looking at it, if you have a sum:\\
\[h(k)=\sum_{i=1}^n c_ip_i^k 1(k)+c_0\]
and $\lvert p_i\rvert <1$ then the sequence is boundable.\\
poles inside the unit circle.\\
mostly going to look at bibo stability in this class, not really interested in state stability. Most of the time it is the same.\\


Jury stability criterion, the digital analogy to Routh Stability criterion.\\

\subsection*{Discrete models of sampled-data systems}
Once again assume that your D to A is a zero order hold device.\\
you can represent your whole system as a discrete transfer function with $u(kT)$ as the input to a D to A and $y(kT)$ as an output from your A to D.\\

state space representation.
\[\Phi, \Gamma, H, J\]
\[x(k+1)=\Phi x(k)+\Gamma u(k)\]
\[y(k)=H x(k)+Ju(k)\]

trying to derive the difference state space equation using differential state space representation.\\

F a matrix, $(e^{F\tau})^{-1}=e^{-F\tau}$.

\[x_p(t)=\int_{t_0}^te^{F(t-\tau)}Gu(\tau)d\tau\]
convolution between Gu and the state transition matrix.\\

\[x(t)=e^{F(t-t_0)}x(t_0)+\int_{t_0}^te^{F(t-\tau)}Gu(\tau)d\tau\]
then sub in $t_0=kT$ and $t=kT+T$. This yields:
\[x(kT+T)=e^{FT}x(kT)+\int_{kT}^{kT+T}e^{F(kT+T-\tau)}Gu(\tau)d\tau\]
with a zero order hold,  assume that $u(\tau)=u(kT)$, $kT\leq \tau<kT+T$ also sub in that $\eta=kT+T-\tau$ and $d\eta=-d\tau$. This gives:
\[-\int_T^0e^{F\eta}Gd\eta u(kT)=\int_0^Te^{F\eta}d\eta Gu(kT)\]
and:
\[x(kT+T)=e^{Ft}x(kT)+\int_0^Te^{F\eta}d\eta Gu(kT)\]
H and J in the continuous time and discrete time model are the same.
\[\Phi=e^{FT}\]
\[\Gamma=\int_0^Te^{F\eta}d\eta G\]
for F and G being A and B in the systems I normally study. Transfer function from state space is:
\[\cfrac{Y(z)}{U(z)}=H(zI-\Phi)^{-1}\Gamma+J\]


next time we will do the same thing we did with the transfer function to discrete domain with just state-space.

\end{document}
