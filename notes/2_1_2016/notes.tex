\documentclass{article}

\usepackage[margin=1in]{geometry}
\usepackage{graphicx}
\usepackage{amsmath}
\usepackage{amsfonts}


\author{Zachary Vogel}
\date{\today}
\title{Notes in ECEN 5448}

\begin{document}
\maketitle


\section*{Missed the first part}
Talking about state space rep of a simple system.\\
State space matrices are F=A, G=B, H=C, and J=D.\\
J will be zero for any strictly proper transfer function.\\

you lazy slut, read the fucking slides cause you aren't paying attention.\\

\section*{Real notess}
a zero of a discrete-time system is a value of z such that the system output y=0 even if the initial state $x(t_0)$ and the forcing input u are nonzero.\\
\[x(k+1)=\Phi x(k)+\Gamma u(k)\]
\[y(k)=Hx(k)+Ju(k)\to\]
\[(zI-\Phi)X(z)-\Gamma U(z)=0\]
\[HX(z)+JU(z)=0\]
\[\begin{bmatrix}zI-\Phi&\Gamma\\H&J\end{bmatrix}\begin{bmatrix}X(z)\\U(z)\end{bmatrix}=0\]
first matrix is $(n+1)\times (n+1)$, second matrix is $(n+1)\times 1$.

\section*{Discrete Time Systems and Dynamic Response}
talking about signal convergence.\\

\[e_4(k)=r^k\cos(k\theta)1(k)\]
\[E_f(z)=\frac{1}{2}\sum_{k=0}^\infty (r^ke^{jk\theta}z^{-k})+r^ke^{-jk\theta}z^{-k}\]
\[=\cfrac{1}{2}\left (\cfrac{1}{1-re^{j\theta}z^{-1}}+\cfrac{1}{1-re^{-j\theta}z^{-1}}\right )\]
ROC $\lvert re^{j\theta}z^{-1}\rvert <1 \to \lvert rz^{-1}\rvert <1$.\\
\[E_4(z)=\cfrac{1}{2}\left (\cfrac{z}{z-re^{j\theta}}+\cfrac{z}{z-re^{-j\theta}}\right )\]
after some work, you get:
\[E_4(z)=\cfrac{z(z-r\cos(\theta))}{z^2-2r(\cos\theta)z+r^2}, \quad \lvert z\rvert >\lvert r\rvert\]
zeros at $0, r\cos \theta$.\\
poles at $\cfrac{2r\cos(\theta)\pm\sqrt{4r^2\cos^2\theta-4r^2}}{2}=r\cos(\theta)\pm jr\sin(\theta)=re^{\pm j\theta}$\\
closer poles are to the origin, faster the decay is.\\

figure 4.24 is not a smith chart.\\


Correspondence of s plane poles and zeros with z plane poles and zeros.
\[Y(s)=\cfrac{s+a}{(s+a)^2+b^2}=\cfrac{s+a}{s^2+2\zeta\omega s+\omega^2}\]
\[y(t)=e^{-at}\cos(bt)1(t)\]
poles: $s_1,s_2=-a\pm jb=-\zeta\omega\pm j\omega\sqrt{1-\zeta^2}$
\[y(k)=y(t)\bigg|_{t=kT}=e^{-aKT}\cos(bkT)1(kT)\]

poles: $z=e^{(-a\pm jb)T}=e^{s_1T}, e^{s_2T}$\\
In general, if a continuous signal $y(t)$ has Laplcae transform $Y(s)$ with poles $s_1,s_2,\dots,$ then the sampled (discrete) signal $y(kT)$ has z-transform $Y(z)$ with poles $z_1,z_2,\dots$ where:
\[z_i=e^{s_iT}\]

\[s_1,s_2=-\zeta\omega T+j\omega\sqrt{1-\zeta^2}\]
only intersted in that till the phase is $\pm \frac{\pi}{T}$ because of aliasing.\\
this all gives log spirals, which are the graph in her notes. Again, they only go to phase=$\pi$ because aliasing produces the rest of the spirtal.

\end{document}
